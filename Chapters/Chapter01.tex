\chapter{Introducción}

En este trabajo se presenta una técnica novedosa de transmisión de datos en redes de tipo broadcast de manera criptográficamente segura utilizando técnicas de espectro expandido.
Los sistemas de comunicaciones ópticos han echo posible las comunicaciones modernas, tecnologías como Internet comunicaciones celulares no son posibles sin una infraestructura optica de comunicaciones de alta velocidad.
Generalmente las redes de alta velocidad son switcheadas y sobre la capa física solo recientemente han sido utilizadas modulaciones y codificaciones mas complejas que un simple Xon-Xoff.
El Backbone ha evolucionado recientemente de 10Gbps, 100Gbps y 400Gbps al momento de escribir este documento, en donde se comienzan utilizan modulaciones de tipo WDM y modulación coherente [CITA]. Estas redes son generalmente switcheadas, en los que un ruteador central procesa electrónicamente los paquetes de datos y los retransmite a los nodos clientes por lo que es un canal dedicado.
Existen redes de tipo broadcast donde existen ventajas como un sistema de ruteo mucho mas simple que puede ser totalmente óptico, pero también poseen varios problemas como tener que compartir el ancho de banda, y problemas de seguridad inherentes al enviar la información a todos los nodos de la red. Este trabajo apunta a solucionar este ultimo problema utilizando tecnicas de tipo CDMA para la transmisión y un código corrector de errores que aprovecha la naturaleza asimétrica de la interferencia en fibra óptica.
