\chapter{Conclusiones}

En esta Tesis se documentó el diseño un sistema de comunicaciones criptográficamente seguro que aprovecha las técnicas de CDMA sobre fibra óptica y sobre ondas sonoras. Durante el transcurso de la investigación, se desarrolló un algoritmo de corrección de errores asimétrico, optimizado para canales de tipo Z.

El resultado es un sistema de red de tipo difusión, capaz de crear múltiples VLANs criptográficamente seguras utilizando cualquier medio de transmisión que pueda ser modelado como un canal Z. 
Se implementaron simuladores en software con el objetivo de obtener estadísticas y mediciones. Luego se realizaron prototipos funcionales tanto sobre software para el caso del medio acústico, como sobre un dispositivo FPGA en el caso del medio de fibra óptica. El protocolo alcanzó una velocidad de 1000 bps con 16 clientes sobre el medio acústico y 5 Gbps con 128 clientes simultáneos con una separación máxima de 20 km, sobre el medio óptico, con una utilización total del medio del 32\%.

Se mostraron resultados tanto de cálculos teóricos, simulaciónes numéricas y mediciones realizadas sobre los prototipos. Los resultados de los cálculos, simulaciones y mediciones fueron consistentes entre sí.

El esquema de transmisión desarrollado puede utilizarse en cualquier medio de transmisión que pueda representarse como un canal Z, como por ejemplo ondas sonoras o acústicas bajo ciertas modulaciones. Debido a esta característica, el mismo protocolo desarrollado con el objetivo de utilizar fibra óptica como medio de transmisión, fue utilizado en una red acústica de baja velocidad entre dispositivos móviles, con un máximo de 16 dispositivos en la red a una distancia de hasta 1.2 metros. Esto abre las puertas a redes ad hoc privadas entre dispositivos, simplificando aplicaciones que hasta ahora necesitaban de una conexión continua a Internet o de tecnologías del tipo NFC.

Concretamente, se presentó el diseño de una red de difusión con un grado de privacidad criptográficamente fuerte. Para ello se utilizó un filtro de Bloom \cite{Bloom70space/timetrade-offs} encriptado, que es utilizado a la vez como el elemento de cifrado y como una primera etapa de corrección de errores. Adicionalmente, se presentó una codificación de datos novedosa\cite{6476559} que incrementa la eficiencia del filtro de Bloom como algoritmo de corrección de errores. El resultado es un protocolo de VLAN capaz de soportar un volumen de información o \textit{throughput} constante sin importar la carga de la red, manteniendo completa privacidad entre sus nodos.

%asfd_hasta_aca

Se demostró la viabilidad del protocolo mediante la implementacion y medición de dos prototipos: el primer prototipo del sistema de comunicaciónes fue realizado sobre fibra óptica con velocidades de transmisión de 5 Gbps. La plataforma de desarrollo es un transceptor laser de comunicaciones del tipo XFP+, y una FPGA del tipo Xilinx Virtex 5.
El segundo prototipo fue implementado exclusivamente en software y utiliza el mismo protocolo pero con diferentes parámetros, esta vez sobre una red encriptada acústica entre dispositivos móviles sin ningún tipo de modificación o hardware adicional.

Esta Tesis apunta a solucionar uno de los problemas de seguridad más graves de las redes de difusión, que es la vulnerabilidad a ataques de espionaje. Por ejemplo, un nodo malicioso en un sistema TDMA puede acceder a la información de cualquier otro nodo, tan sólo escuchando en el tiempo asignado al nodo víctima, ya que los tiempos de bit son totalmente predecibles.  

En contraste, el sistema presentado en esta Tesis asigna los tiempos de bit de manera pseudoaleatoria, por lo que un nodo malicioso no puede predecir la posición de ningún otro nodo. No existe ningún tipo de arbitraje ni de colaboración entre nodos que pudiera revelar información a un presunto atacante, y se garantiza la comunicación privada de un nodo aún cuando todos los demás nodos del sistema sean maliciosos.

La fuerza criptográfica del sistema esta asociada y es equivalente a la del generador pseudoaleatorio seleccionado para generar los tiempos de bit, siendo el único requerimiento del mismo que sea criptográficamente seguro. Al no existir colaboración y arbitraje entre nodos, la naturaleza o algoritmo de dicho generador puede variar de nodo a nodo sin afectar la performance del sistema. 

La consecuencia de prohibir toda colaboración o arbitraje, es que las colisiones de datos entre nodos son inevitables, y frecuentes. De ahí que el mayor esfuerzo de diseño en el del sistema presentado, y el módulo de mayor consumo de recursos computacionales, es la recuperación o corrección de errores, que debe ser suficientemente potente como para reducir las tasas de error a valores utilizables, sin consumir excesivos recursos computacionales o de ancho de banda. El algoritmo final fue medido con un BER menor a 10e-8 con una utilización del medio del 32\%. Otra característica notable es que, al ser los nodos totalmente independientes entre sí, no se produce ningún tipo de degradación de la velocidad o ancho de banda en los canales de transmisión individuales, siempre y cuando la cantidad total de nodos este por debajo del límite diseñado.

\section{Trabajos futuros}

Todos los prototipos realizados son completamente funcionales y cumplen con los objetivos de la Tesis sin embargo, para una implementación comercial a gran escala es posible optimizar ciertos módulos, que serán nombrados a continuación:
\begin{description}
 \item[Codificación] 
El problema de la codificación de línea fue descrito en \ref{problemacodificacion} y una solución aceptable, que permite la creación de un prototipo funcional, fue adoptada. Sin embargo, en una implementación final, se debe contemplar la imposibilidad de utilizar cualquier algoritmo de balanceo ya que no es compatible con el algoritmo de filtro de Bloom. Dado que el problema es de naturaleza eléctrica, es probable que un diseño cuidadoso de los buses entre la FPGA y el transceptor óptico solucione este problema.
 \item[Sincronización] 
 El algoritmo de sincronización es crítico en una implementación real. Las implementaciones realizadas para los prototipos son funcionales pero no son ideales ya que, al utilizar prefijos conocidos, un atacante puede obtener información acerca del comienzo y finalización de la trama. La codificación es suficientemente fuerte para que este conocimiento no perjudique el nivel de seguridad, pero es posible el desarrollo de un algoritmo de sincronización criptográficamente seguro, que no revele ninguna información a un posible atacante (ver \cite{jung1999encryption} para un ejemplo de sincronización segura).
 \item[Eliminación de la trama] 
 En la sección \ref{Seguridad} se explica que se divide la transmisión en tramas, que son segmentos de bits cargados directamente dentro del filtro de Bloom. Esta división en segmentos de los datos no es absolutamente necesaria y podría implementarse una variación del algoritmo que no utilice tramas, sino que todas las posiciones de bits sean relativas entre sí, en lugar de relativas al comienzo de la trama. Esta variación no incrementa la seguridad pero, probablemente, simplifique la implementación.
 \item[] 
\end{description}

